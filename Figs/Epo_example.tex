\documentclass[hidelinks,11pt]{article}
\usepackage[top=0.7in,left=0.75in,right=0.75in,footskip=0.7in]{geometry}
\usepackage{amsmath,amssymb}
\usepackage[utf8x]{inputenc}
\usepackage{cite}
\usepackage{hyperref}
\usepackage{multicol}
\usepackage[switch]{lineno}
\usepackage[table]{xcolor}
\usepackage{array}
\usepackage{multirow}
\usepackage{rotating}
\newcommand\tabrotate[1]{\begin{turn}{90}\rlap{#1}\end{turn}}
\usepackage{subscript}


\renewcommand{\vec}[1]{{\mbox{\mathversion{bold}\ensuremath{#1}}}}

\usepackage[aboveskip=1pt,labelfont=bf,labelsep=period,singlelinecheck=off]{caption}

\bibliographystyle{vancouver}

\date{}
\usepackage{lastpage,fancyhdr,graphicx}
\usepackage{epstopdf}
\usepackage[normalem]{ulem}
\newcommand{\benny}[1]{\textcolor{blue}{{#1}}}
\newcommand{\maik}[1]{\textcolor{green}{{#1}}}
\newcommand{\rem}[1]{\textcolor{red}{\sout{#1}}}

\pagestyle{myheadings}
\pagestyle{fancy}
\fancyhf{}
\setlength{\headheight}{27.023pt}
\rfoot{\thepage/\pageref{LastPage}}
\renewcommand{\footrule}{\hrule height 2pt \vspace{2mm}}
\addtolength{\columnwidth}{0.34in}
%%%%%%%%%%%%%%%%%%%%%%%%%%%%%%  Own commands



\begin{document}


\twocolumn[
  \begin{@twocolumnfalse}

\begin{flushleft}
{\Large\textbf{Data Assimilation for Systems and Mathematical Biology} 
}
\newline

Engelhardt Benjamin	\textsuperscript{1,2,*},
Kahl Dominik\textsuperscript{3},
Weber Andreas\textsuperscript{4},
Kschischo Maik\textsuperscript{3,*}
\\
\bigskip
\textbf{1} Rheinische Friedrich-Wilhelms-Universit\"at Bonn, Algorithmic Bioinformatics, Bonn, Germany
\\
\textbf{2} Current Address: AbbVie Deutschland GmbH {\&} Co. KG, Ludwigshafen, Germany
\\
\textbf{3} University of Applied Sciences Koblenz, RheinAhrCampus, Department
of Mathematics and Technology, Remagen, Germany
\\
\textbf{4} Rheinische Friedrich-Wilhelms-Universit\"at Bonn, Institute for Computer Science, Bonn, Germany
\\
 \bigskip
*Authors for correspondence: engelhar@bit.uni-bonn.de and kschischo@rheinahrcampus.de

\end{flushleft}

\section*{Abstract}
Mathematical models are increasingly used as a tool to deal with
the tremendous complexity of biological systems. Data Assimilation, defined as the process of combining models with experimental observations, is a key step in order to better align the model outputs with reality. Despite new and improved experimental techniques, it is usually impossible to directly observe all the states of a biological system. This renders Data Assimilation an inverse problem requiring sophisticated mathematical and statistical techniques  and the systematic integration of prior knowledge or assumptions.
\vspace*{2cm}
  \end{@twocolumnfalse}
]

\section{Introduction}
\section{The Data Assimilation problem}

\section{Outlook} 

\begin{align*}	
\dot{\vec{x}} =&  \underbrace{\left(\begin{array}{c}
k_{t} B_{max} - k_t x_1 - k_{on} x_1 x_2+k_{off} x_3+k_{ex} x_4 \\ 
-k_{on} x_1 x_2+k_{off} x_3+k_{ex} x_4\\ 
k_{on} x_1 x_2-k_{off} x_3- k_e x_3\\ 
k_e x_3-k_{ex} x_4-k_{di} x_4-k_{de} x_4\\ 
k_{di} x_4\\ 
k_{de} x_4 \\ 
\end{array} \right)}_{\vec{f} \left( \vec{x} \right)}\\
%
\vec{y} 		  =& \underbrace{\left(\begin{array}{c}
\kappa_1 \left({x}_2 +2{x}_6\right)\\
\kappa_2\left({x}_3\right)\\
\kappa_3\left({x}_4 +{x}_5\right)
\end{array} \right)}_{\vec{h}\left(\vec{x}\right)}
\end{align*}
%

\begin{align*}	
x_1:& \,\text{EpoR}\\
x_2:& \,\text{Epo}\\
x_3:& \,\text{Epo-EpoR}\\
x_4:& \,\text{Epo-EpoR}_i\\
x_5:& \,\text{dEpo}_i \\
x_6:& \,\text{dEpo}_e 
\end{align*}

\begin{align*}	
y_1: & \,\text{Epo + dEpo}_i\\
y_2: & \,\text{Epo-EpoR} \\
y_3: & \,\text{Epo-EpoR}_i + \text{dEpo}_i\\
\end{align*}


\begin{tabular}{|c|c|c|c|c|c|}
\hline 
$x_1$ & $x_2$ & $x_3$ & $x_4$ & $x_5$ & $x_6$ \\ 
\hline 
$\text{EpoR}$ & $\text{Epo}$  & $\text{Epo-EpoR}$ & $\text{Epo-EpoR}_i$ & $\text{dEpo}_i$&  $\text{dEpo}_e$ \\ 
\hline 
\end{tabular} 

\vspace{1cm}

\begin{tabular}{|c|c|c|}
\hline 
$y_1$ & $y_2$ & $y_3$\\ 
\hline 
$\text{Epo + dEpo}_i$ & $\text{Epo-EpoR}$ & $\text{Epo-EpoR}_i + \text{dEpo}_i$ \\ 
\hline 
\end{tabular} 




\newpage\newpage


To illustrate different aspects of DA we will use a model for the information processing at the erythropoietin (Epo) receptor (EpoR) as a running example~\cite{becker_covering_2010}. The state $\vec{x}=(x_1,\ldots,x_6)^T$ of this model is given by the concentrations of the Epo receptor ($x_1$) on the cell surface 
which can bind to Epo ($x_2$) and build the ligand-receptor complex ($x_3$). This complex is able to activate subsequent signaling cascades, e.g. the JAK-STAT signaling pathway. In addition the ligand-receptor complex can be internalized ($x_4$) and dissociate from Epo which then is degraded ($x_5$) and transported to the extracellular space ($x_5$). 
\begin{eqnarray*}
\dot{{x}}_1 &=& k_{t} B_{max} - k_t x_1 - k_{on} x_1 x_2+k_{off} x_3+k_{ex} x_4   \\      
\dot{{x}}_2 &=& -k_{on} x_1 x_2+k_{off} x_3+k_{ex} x_4   \\                          
\dot{{x}}_3 &=&k_{on} x_1 x_2-k_{off} x_3- k_e x_3      \\                          
\dot{{x}}_4 &=& k_e x_3-k_{ex} x_4-k_{di} x_4-k_{de} x_4  \\                          
\dot{{x}}_5 &=& k_{di} x_4                               \\                           
\dot{{x}}_6 &=& k_{de} x_4                             \\                             
		{y}_2&=&\kappa_1 \left({x}_2 +2{x}_6\right)\\
		{y}_1&=&\kappa_2\left({x}_3\right)\\
		{y}_3&=&\kappa_3\left({x}_4 +{x}_5\right),		
\end{eqnarray*}



The complex regulation of this receptor is characterized by receptor mobilization, turnover and recycling. The rate constants correspond to (i) receptor turnover ($k_t$), (ii) ligand-receptor binding ($k_{on}$) or dissociation ($k_{off}$), (iii) ligand-induced endocytosis ($k_e$), (iv) recycling ($k_{ex}$) and (v) internal ($k_{di}$) or external ($k_{de}$) degradation of Epo. Only the the Epo concentration in medium ($y_1$), on surface ($y_2$) and in cells ($y_3$) can be measured up to some scaling paramters $\kappa_j,\, j\in \{1,2,3\}$.  

\twocolumn[
  \begin{@twocolumnfalse}
\begin{align*}	
\dot{\vec{x}} =&  \underbrace{\left(\begin{array}{c}
k_{t} B_{max} - k_t x_1 - k_{on} x_1 x_2+k_{off} x_3+k_{ex} x_4 \\ 
-k_{on} x_1 x_2+k_{off} x_3+k_{ex} x_4\\ 
k_{on} x_1 x_2-k_{off} x_3- k_e x_3\\ 
k_e x_3-k_{ex} x_4-k_{di} x_4-k_{de} x_4\\ 
k_{di} x_4\\ 
k_{de} x_4 \\ 
\end{array} \right)}_{\vec{f} \left( \vec{x}, \vec{u} \right)}
\begin{array}{l}
\text{EpoR}\\
\text{Epo}\\
\text{Epo-EpoR}\\
\text{Epo-EpoR}_i\\
\text{dEpo}_i \\
\text{dEpo}_e 
\end{array} \\
%
\vec{y} 		  =& \underbrace{\left(\begin{array}{c}
\kappa_1 \left({x}_2 +2{x}_6\right)\\
\kappa_2\left({x}_3\right)\\
\kappa_3\left({x}_4 +{x}_5\right)
\end{array} \right)}_{\vec{h}\left(\vec{x}\right)}
\begin{array}{l}
\text{Epo + dEpo}_i\\
\text{Epo-EpoR} \\
\text{Epo-EpoR}_i + \text{dEpo}_i\\
\end{array}\\
\end{align*}

\end{@twocolumnfalse}
]

\appendix

\section{Appendix name}


%\section*{References}
\bibliographystyle{vancouver}
%\addcontentsline{toc}{section}{\refname}
\bibliography{RefsDAinSysMathBiol}


\end{document}